\section{Polar-Diagramm}
\subsection{Parser}
Die Schiffsdaten sind als CSV\footnote{Comma Separated Values}-Datei
gegeben und werden von einem selbst\-geschriebenen Parser in ein
passendes Klassen-Konstrukt geladen. Die Schwierigkeit bestand darin,
dass die Daten nicht-dekoriert sind und daher nur durch ihre Position
von anderen Typen zu unterscheiden sind. Das bedeutet allerdings auch,
dass der Parser sich auf diese Datenstruktur verlässt und keine
Abweichungen duldet und sonst abstürzt.

Dieses Problem ist allerdings vertretbar, da die Datenstruktur
vorgegeben ist und der Parser daran angepasst wurde.

\subsection{Verarbeitungsklasse}
Die Klasse \texttt{BoatSpeedDiagram.java\footnote{Im Package 
ch.zhaw.lakerouting.interpolation.boatdiagram zu finden}} verwaltet das
Polardiagram und bietet die nötigen Methoden für die Abfrage der Werte
an, kapselt die Interpolation, so das der Programmierer sich darum nicht
kümmern muss.

\subsection{Bilineare Interpolation}\label{sss:bilinearinterpolation}
Die Interpolation an sich ist durch 
\begin{equation}
f(x,y) \approx \begin{bmatrix} 1-x & x \end{bmatrix} \begin{bmatrix}
f(0,0) & f(0,1) \\ f(1,0) & f(1,1) \end{bmatrix} \begin{bmatrix} 1 - y
\\ y \end{bmatrix}
\label{eq:bilineareinterpolation}
\end{equation}
geben und kann in Java wie nachfolgend gezeigt, implementiert werden.

 \lstinputlisting[label=src:bilinearinterpolation,caption=Bilineare Interpolation]{code/BilinearInterpolation.java}
Die konkrete Berechnung erfolgt auf den Zeilen 17-19 und zeigen den
einfachen Charakter der Berechnung. Diese Implementation erfordert
allerdings eine geringe Aufbereitung der Eingabewerte, ist dafür
einfacher zu testen und somit indirekt auch weniger anfällig für
Implementationsfehler, was direkt zu robusterem Code führt.

\section{Windfelder}
Die Implementation ist vergleichsweise einfach gehalten und besteht aus
einer Klasse für ein einzelnes Windfeld mit Hilfsfunktionalitäten für
die Interpolation der Windvektoren und das Abfragen der Nachbarsvektoren
an einer bestimmten Position. Diese einzelnen Windfelder werden dann in
einem Art Stapel aufbewahrt\footnote{Achtung, nicht mit einem Stack
verwechseln! Wir verwenden hier nichts anderes als ein Array.} Im
Prinzip besteht das alles aus einer dreidimensionalen Struktur aus
Windvektoren.

\subsection{Windfeld-Parser}
Der Parser ist ein Eigenbau und wurde speziell für diese Datenstruktur
geschaffen. Diverse Unzulänglichkeiten wie nicht-leere Zeilen und
Trailing Spaces haben die Entwicklung erschwert und dementsprechend viel
Zeit gekostet.

Allerdings lässt sich der Parser leicht auf andere Ressourcen erweitern
wie zum Beispiel HTTP oder auch direkte Anbindung an Datenbanken.

\subsection{Interpolation der Windvektoren}
Da das Windfeld und das Entscheidungsnetz nicht über dieselben
Koordinaten verfügen und dementsprechend die Entscheidungspunkte
normalerweise zwischen den Windvektoren zu liegen kommen, wird nach
Berechnung des Entscheidungsnetzes dem Windfeld die Koordinatenliste
übergeben und dieser interpoliert die Windrichtung bzw. die Windvektoren
an den geforderten Positionen. Das erfolgt wie bereits in Abschnitt
\ref{sss:bilinearinterpolation} erklärt über dieselbe Klasse.

Nach dieser initialen Interpolation verfügt das Entscheidungsnetz und
das Windfeld über dieselben Array-Indizes und können somit leicht
(wieder-) verwendet werden.
