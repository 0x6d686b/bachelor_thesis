% Author: Mathias Hablützel

\section{Berechnung von Segelgeschwindigkeit anhand der Windfelder}
Die in Abschnitt \ref{s:windfields} eingelesenen Windfelder werden nun
für die Berechnung der Segelgeschwindigkeit verwendet. Gegen sind der
Windangriffswinkel auf das Schiff durch die Windvektoren von Punkt $A$
nach $B$, die Segelrichtung ist damit auch bekannt.

% TODO: Skizze zeichnen mit Punkt A -> B, Segelboot, Windvektoren
 
\subsection{Geschwindigkeit am Ausgangspunkt}
Da es bekannt ist zu welcher Zeit man am Ausgangspunkt $A$ ist, weiss
man auch welcher Windvektor existiert (der sich ja mit der Zeit
verändert). Somit lässt sich die Geschwindigkeit des Bootes am
Ausgangspunkt berechnen.

\subsection{Geschwindigkeit am Zielpunkt}
Hier besteht das Problem, dass es nicht bekannt ist, wann das Boot am
Punkt $B$ ankommt, die Ankunftszeit hängt vom Windvektor $v_{1}$ in
Punkt $A$ zur Zeit $t_{1}$ und vom Windvektor $v_{2}$ in Punkt $B$ zur
Zeit $t_{2}$ ab. $t_{2}$ ist unbekannt und folglich ist $v_{2}$ auch
nicht bekannt (nur dessen Position ist im Voraus bekannt).

Also muss in einem ersten Schritt die Ankunftszeit approximiert werden.
Wir könnten die Euler'sche Regression verwenden, allerdings besteht hier
das Problem, dass der Algorithmus nicht zwingend die beste Lösung findet
und über eine unterschiedlich lange Laufzeit verfügt. Daher erschien ein
zweischrittiges Verfahren für eine Teilapproximation vernünftig, welches in
linearer Zeit ausgeführt werden kann.

\subsection{Berechnung der Ankunftszeit}
Vom Startpunkt $A$ ist die Uhrzeit, die Position und somit der Windvektor
bekannt. Daraus kann die Geschwindigkeit des Segelbootes bestimmt werden
(ausgehend von Punkt $A$). Diese Geschwindigkeit wird nun für die gesamte
Distanz angenommen und daraus resultiert eine ungefähre Ankunftszeit. Wir
nehmen hier an, dass das Windfeld zu einem späteren Zeitpunkt nicht so stark
ändert, dass unsere Berechnungen zu stark von einem vernünftigen Wert abweicht,
unter anderem weil dies der Natur von Windfeldern widersprechen würde.
Ausgenommen sind Extremsituationen wie Windhosen oder Tornados. Aber in solchen
Situationen wäre das Segeln auch nicht vernünftig.

\subsection{Implementation}
In der Implementation haben wir einen noch einfacheren Ansatz verfolgt:

\begin{align}
t = \frac{1}{2} (\frac{2d}{|\mathbf{\overrightarrow{v_{1}}} + \mathbf{\overrightarrow{v_{2}}}|})
\end{align}

t Reisezeit, d Distanz zwischen zwei Punkten, $v_{1}$ bzw. $v_{2}$
Windvektoren an den respektiven Koordinaten.

\vspace{0.5cm}

Würde der Windvektor $v_{2}$ aufgrund von der absoluten Zeit, also in
Bezug zur Uhrzeit, auf das Windfeld des nächsten Zeitabschnitts fallen,
so wird $v_{2}$ mit dem des (zeitlich) nächsten Windfeldes ersetzt.
