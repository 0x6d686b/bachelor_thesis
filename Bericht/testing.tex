\section{Testing}
Um die Applikation auf Erfüllung der für ihren Einsatz definierten Anforderungen zu bewerten und zu prüfen sind Softwaretests nötig. 

\subsection{JUnit Test}
JUnit ist ein kleines, mächtiges Java-Framework zum Schreiben und Ausführen von automatisierten Unit-Tests. Es werden in der Regel für alle Klassen, die überprüft werden sollen, ein Unit-Test erstellt. Diese Unit-Tests verhalten sich auch wie die gewöhnlichen Java-Klassen, mit dem Unterschied, dass sie im Namen das Wort $Test$ am Schluss und einige JUnit-Annotationen innerhalb der Klasse haben, wobei die Endung im Namen nur eine Konvention ist. Ein JUnit-Test kennt nur zwei Ergebnisse: Entweder der Test war erfolgreich oder nicht. Die geschriebenen Testfälle sind zu jederzeit wiederholbar. 

Diese Softwaretests wurden in zwei Arten durchgeführt: Erstens als Unit-Tests und zweitens als manuelle Tests. Überall wo die Unit-Tests möglich waren, wurden sie auch implementiert.
\subsection{Beispiele}
% Mit sinnvollen Beispielen simulieren.