\section{Testing}
Um die Applikation auf Erfüllung der für ihren Einsatz definierten
Anforderungen zu bewerten und zu prüfen sind Softwaretests nötig. Es wurde
versucht, diese Tests am Anfang zu erstellen und im Laufe der Entwicklung
regelmässig zu durchführen und erweitern. Es wurden in der Regel zwei Arten
von Tests eingesetzt: manuelle und automatisierte Tests. Die manuellen Tests
wurden einerseits mit dem Debugger zur Verifizierung der Inhalte und
andererseits mit systematisch festgelegten Eingabedaten ausgeführt. Daneben
wurden auch noch die automatisierten Unit-Tests eingesetzt, um zu
verifizieren, dass bei Änderungen keine unerwünschten Nebeneffekte
hervortreten. 

In unserer Appkilation wurden für alle Packages mindestens ein JUnit-Test
geschrieben. Es wurden entweder ganze Klassen oder nur gewisse, essenzielle
Methoden getestet. Somit sollten diese Tests die meisten Fällen abdecken. 

\subsection{JUnit Test}
JUnit ist ein kleines, mächtiges Java-Framework zum Schreiben und Ausführen
von automatisierten Unit-Tests. Es werden in der Regel für alle Klassen, die
überprüft werden sollen, ein Unit-Test erstellt. Diese Unit-Tests verhalten
sich auch wie die gewöhnlichen Java-Klassen, mit dem Unterschied, dass sie im
Namen das Wort \texttt{Test} am Schluss und einige JUnit-Annotationen innerhalb der
Klasse haben, wobei die Endung im Namen nur eine Konvention ist. Ein
JUnit-Test kennt nur zwei Ergebnisse: Entweder war der Test erfolgreich oder
nicht. Das Fehlschlagen kann zwei Gründe haben, entweder war es ein
Fehler(Error) oder es lieferte ein falsches Ergebnis (Failure). Der einzige
Unterschied zwischen den beiden Begriffen liegt darin, dass Errors eher
unerwartet auftreten, während Failures erwartet werden. Desweiteren können die
geschriebenen Testfälle zu jeder Zeit beliebig wiederholt werden. 

Ausserdem bietet dieses Framework einige Methoden an, um Werte vergleichen zu
können. So zum Beispiel vergleicht die Methode \texttt{assertEquals(expected, actual,
delta)} zwei Werte, expected und actual, und kann eine Abweichung beinhalten. 

\subsubsection{Beispiele}
% Mit sinnvollen Beispielen simulieren.

