\section{Testing}
Um die Applikation auf Erfüllung der für ihren Einsatz definierten Anforderungen zu bewerten und zu prüfen sind Softwaretests nötig. Deswegen haben wir versucht, von Anfang an die Tests zu schreiben und wo immer es möglich war, die Applikation während der Implementierung zu testen. 

\subsection{JUnit Test}
JUnit ist ein kleines, mächtiges Java-Framework zum Schreiben und Ausführen von automatisierten Unit-Tests. Es werden in der Regel für alle Klassen, die überprüft werden sollen, ein Unit-Test erstellt. Diese Unit-Tests verhalten sich auch wie die gewöhnlichen Java-Klassen, mit dem Unterschied, dass sie im Namen das Wort $Test$ am Schluss und einige JUnit-Annotationen innerhalb der Klasse haben, wobei die Endung im Namen nur eine Konvention ist. Ein JUnit-Test kennt nur zwei Ergebnisse: Entweder war der Test erfolgreich oder nicht. Das Fehlschlagen kann zwei Gründe haben, entweder war es ein Fehler(Error) oder es lieferte ein falsches Ergebnis (Failure). Der einzige Unterschied zwischen den beiden Begriffen liegt darin, dass Errors eher unerwartet auftreten, während Failures erwartet werden. Desweiteren können die geschriebenen Testfälle zu jeder Zeit beliebig wiederholt werden. 

\subsubsection{Beispiele}
% Mit sinnvollen Beispielen simulieren.
